\documentclass[10pt]{scrreprt}

\usepackage[utf8]{inputenc}
\usepackage[ngerman]{babel}
\usepackage{xspace}

\author{Matthias Werner}
\title{Die \LaTeX-Klasse \texttt{osgbeamer}}
\subject{Dokumentation}

\newcommand{\osg}{\texttt{osgbeamer}\xspace}
\date{}
\begin{document}
\maketitle

\chapter{Einführung}
Die \LaTeX-Klasse \osg dient dazu, verschiedene verschiedensprachliche Unterstützungsmaterialien für
Lehrveranstaltungen an der Professur Betriebssysteme der TU~Chemnitz aus dem
gleichen \LaTeX-Quellfile zu generieren.
Konkret können folgende Materialien erzeugt werden:
\begin{itemize}
  \item Vorlesungsfolien
  \item Handouts (d.h. die Folien im 2$\times$2-Format)
  \item Skript-Kapitel
  \item Präsentationen mit (nur für den Präsentierenden sichtbaren)
  Anmerkungen\footnote{derzeit nicht implementiert}
\end{itemize}
Es können deutsche und englische Dokumente erstellt werden; andere Sprachen
sind derzeit nicht vorgesehen.

\section{Voraussetzungen}
\osg wurde für eine aktuelle \TeX Live Installation auf macOS entwickelt und
ist möglicherweise auch auf anderen \TeX-Distributionen lauffähig. 
\end{document}
